% !TeX root = README.tex

\section{Getting Started Guide}
\label{sec:getting-started-guide}

Docker is required to run the artifact.
%
First, install Docker Engine on your machine as instructed in
\url{https://docs.docker.com/engine/install/}.
%
To check if Docker has been installed properly, run
\begin{verbatim}
$ docker --version
Docker version 28.1.1, build 4eba377
\end{verbatim}

Load the Docker image \texttt{resource-decomposition.tar.gz} by running
\begin{verbatim}
$ docker load --input resource-decomposition.tar.gz
\end{verbatim}
%
It creates an image named \texttt{resource-decomposition} and stores it locally
on your machine.
%
Docker may create an image with a slightly different name from
\texttt{resource-decomposition}.
%
To check the name of the image, display all Docker images on your local machine
by running
\begin{verbatim}
$ docker images
\end{verbatim}

To run the image \texttt{resource-decomposition}, run
\begin{verbatim}
$ docker run --name resource-decomposition -it --rm resource-decomposition
root@25bfacdb517e:/home/ocaml-benchmarks#
\end{verbatim}
%
It creates a Docker container (i.e., a runnable instance of the Docker image),
which has the same name \texttt{resource-decomposition} as the Docker image.
%
The command also starts a shell under the directory
\texttt{/home/ocaml-benchmarks}.
%
If the command does not run properly, you can instead build the image locally on
your machine as instructed in \cref{sec:reusability-guide}.

Throughout this document, any command line starting with \texttt{\#} is executed
inside the Docker container, and any command line starting with \texttt{\$} is
executed in your local machine's terminal.

\subsection{Demonstration of Resource Decomposition's First Instantiation}

This section demonstrates the first instantiation of resource decomposition
(\labelcref{introduction:instantiation:1}), which integrates static and Bayesian
data-driven resource analyses.
%
The static part runs a program-analysis tool RaML~\citep{RaML,Hoffmann2017}.
%
It implements type-based static resource analysis Automatic Amortized Resource
Analysis (AARA)~\citep{Hofmann2003,Hoffmann2011a,Hoffmann2022} and automatically
infers polynomial cost bounds of input OCaml programs.
%
The data-driven part, on the other hand, runs Bayesian inference to
statistically infer a symbolic bound of a resource component using its
measurements.

For illustration, let us consider the benchmark \mergesort{} implemented in
OCaml.
%
The resource metric of interest is the number of function calls (including all
recursive calls and helper functions).
%
Our goal is to infer an asymptotically tight $O (n \log n)$ (with concrete
coefficients) cost bound of \mergesort{}.

\paragraph{Original program}

Inside the Docker container, the current working directory should be
\begin{verbatim}
# pwd
/home/ocaml-benchmarks
\end{verbatim}
%
Let $P(x)$ denote an original program of \mergesort{}.
%
To view its source code, run
\begin{verbatim}
# cat lib/merge_sort/merge_sort.ml
\end{verbatim}
%
This OCaml code is annotated with the construct \texttt{Raml.tick 1.0} to
specify resource usage: it increments a cost counter by 1.0.
%
This construct is inserted at the start of every function.

\paragraph{Static analysis on the original program}

RaML can infer a polynomial bound for \mergesort{}.
%
To invoke RaML on \mergesort{}, run a python script:
\begin{verbatim}
# python3 raml/run_raml.py benchmark merge_sort standard
\end{verbatim}
%
It prints out the inference result of static resource analysis:
\begin{Verbatim}[fontsize=\footnotesize]
  == merge_sort :

  int list -> int list

  Non-zero annotations of the argument:
  7  <--  [::(*); ::(*)]
  1  <--  [::(*)]
  1  <--  []

  Non-zero annotations of result:

  Simplified bound:
  1 - 2.5*M + 3.5*M^2
  where
  M is the number of ::-nodes of the argument

  --
  Mode:          upper
  Metric:        ticks
  Degree:        2
  Run time:      0.13 seconds
  #Constraints:  777

  ====
\end{Verbatim}
%
In the middle of the displayed inference result, we can find a quadratic cost
bound of \mergesort{} inferred by RaML:
\begin{equation}
  1 - 2.5 n + 3.5 n^2,
\end{equation}
where $n$ is the input size.
%
Although this quadratic bound is sound (i.e., it is a valid worst-case cost
bound for all inputs), it is not an asymptotically tight $O(n \log n)$ bound of
\mergesort{}.

\paragraph{Resource-decomposed and resource-guarded programs}

To obtain a tighter cost bound of \mergesort{}, we integrate AARA and Bayesian
data-driven analysis via resource decomposition.
%
We set a resource component to be the recursion depth of the function
\texttt{merge\_sort} in the original program $P(x)$.
%
To specify this resource component, the original program $P(x)$ is annotated
with the construct \texttt{Raml.mark}, resulting in a resource-decomposed
program $P_{\mathrm{rd}}(x)$.
%
To view its source code, run
\begin{verbatim}
# cat lib/merge_sort/merge_sort_data_collection.ml
\end{verbatim}
%
A recursion-depth bound of the function \texttt{merge\_sort} will later be
inferred by Bayesian data-driven analysis.

The original program $P(x)$ is transformed to a resource-guarded program
$P_{\mathrm{rg}}(x, r)$, which has two inputs: the original input $x$ and a
resource guard $r$.
%
The resource guard $r$ is a numeric variable that tracks the recursion depth of
the function \texttt{merge\_sort}.
%
To view the code of the resource-guarded program $P_{\mathrm{rg}}(x, r)$, run
\begin{verbatim}
# cat lib/merge_sort/merge_sort_counter.ml
\end{verbatim}
%
The resource-guarded program $P_{\mathrm{rg}}(x, r)$ is obtained from the
original program $P(x)$ by manually extending it with a resource guard.

\paragraph{Static analysis on the resource-guarded program}

To infer an overall cost bound of the resource-guarded program
$P_{\mathrm{rg}}(x, r)$, invoke RaML by running
\begin{verbatim}
# python3 raml/run_raml.py benchmark merge_sort counter
\end{verbatim}
%
It prints out the inference result:
\begin{Verbatim}[fontsize=\footnotesize]
  == merge_sort :

  [int list; int list * 'a] -> int list * (int list * 'a)

  Non-zero annotations of the argument:
  3.5  <--  ([::(*)], ([::(*)], *))
  1  <--  ([], ([], *))

  Non-zero annotations of result:

  Simplified bound:
  1 + 3.5*L*M
  where
  L is the number of ::-nodes of the 1st component of the 2nd component of the argument
  M is the number of ::-nodes of the 1st component of the argument

  --
  Mode:          upper
  Metric:        ticks
  Degree:        2
  Run time:      0.28 seconds
  #Constraints:  1192

  ====
\end{Verbatim}
%
Here, an inferred cost bound is
\begin{equation}
  1 + 3.5 n \cdot r,
\end{equation}
where $n$ is the length of the original input list and $r$ is the value of the
resource guard.
%
Also, the analysis time by RaML is \qty{0.28}{\second}.
%
In the last column of Table~1 in the paper, the reported analysis time for
\mergesort{} (the first row) is \qty{0.4}{\second}.
%
On your local machine, RaML's analysis time should not deviate significantly
from this reported analysis time.

\paragraph{Bayesian analysis on the resource-decomposed program}

We perform Bayesian data-driven resource analysis on the resource-decomposed
program $P_{\mathrm{rd}}(x)$ to infer a recursion-depth bound of the function
\texttt{merge\_sort}.

First, switch the working directory to \texttt{/home/experiment}:
\begin{verbatim}
# cd ../experiment
\end{verbatim}
%
We run the resource-decomposed program $P_{\mathrm{rd}}(x)$ on many input lists
to create a dataset $\calD$ of recursion depths' measurements.
%
To this end, run
\begin{verbatim}
# python3 -m runtime_data_generation.runtime_data_generation benchmark merge_sort
\end{verbatim}
%
This command randomly generates many integer lists (of varying lengths), where
integers are drawn uniformly at random from a broad interval.
%
The command then executes the OCaml resource-decomposed program
$P_{\mathrm{rd}}(x)$ of \mergesort{} on these lists, recording the input sizes
and recursion depths.
%
The generated dataset $\calD$ is stored in this JSON file:
\begin{verbatim}
/home/experiment/bin/merge_sort/bucket1/runtime_cost_data/runtime_cost_data.json
\end{verbatim}

We next perform Bayesian inference by running
\begin{verbatim}
# python3 -m run_model benchmark merge_sort
\end{verbatim}
%
This command invokes the inference engine of the probabilistic programming
language Stan~\citep{Carpenter2017}.
%
It runs a sampling-based probabilistic inference algorithm, drawing a large
number (specifically 44,000) of posterior recursion-depth bounds of
\mergesort{}.

The command takes approximately \qty{45}{\second} to run for the first time.
%
The bulk of the analysis time is spent in building efficient machine code for
Bayesian inference---Bayesian inference itself only takes around
\qty{4.5}{\second}.
%
The posterior samples drawn by a sampling algorithm (and other information such
as its analysis time) are stored in this JSON file:
\begin{verbatim}
/home/experiment/bin/merge_sort/bucket1/counter0/model0/inference_result.json
\end{verbatim}

\paragraph{Display analysis time}

To print out Bayesian inference's analysis time, run
\begin{verbatim}
# python3 -m run_analysis time
\end{verbatim}
%
It shows a table of Bayesian inference's analysis time for all 13 benchmarks in
\labelcref{introduction:instantiation:1}, including \mergesort{}, and also the
benchmark \quicksorttiml{} in \labelcref{introduction:instantiation:3}.
%
The table should look like
\begin{verbatim}
Benchmark                     Counter Analysis Time (s) Iterations Chains
merge_sort                          0              4.45      11000      4
...
\end{verbatim}

In the paper, Bayesian inference's analysis time is reported in the
second-to-last column of Table~1.
%
On your local machine, the analysis time for \mergesort{} should not deviate
significantly from the analysis time reported in the paper, namely
\qty{6.0}{\second}.

\paragraph{Display soundness percentages}

To print out the percentages of sound recursion-depth bounds inferred by
Bayesian inference, run
\begin{verbatim}
# python3 -m run_analysis soundness all
\end{verbatim}
%
The command prints out a table of soundness percentages for all 13 benchmarks in
\labelcref{introduction:instantiation:1} and the benchmark \quicksorttiml{} in
\labelcref{introduction:instantiation:3}.
%
The table should look like
\begin{verbatim}
Benchmark                     Conv. AARA   Counter Sound Complexity  Sound Bounds
merge_sort                    Wrong Degree       0           100.0%         56.1%
...
\end{verbatim}

In the paper, the soundness percentages are reported in the third-to-last and
fourth-to-last columns of Table~1.
%
On your local machine, you should get the exact soundness percentages of
\mergesort{} of 100\% (for asymptotic complexity) and  56.1\% (for coefficients)
because we have fixed the seeds of all pseudo-random number generators used in
the artifact.

The soundness percentage 56.1\% (for coefficients) returned by the artifact
deviates from the percentage reported in Table~1 in the paper, namely 59.5\%.
%
We suspect that this deviation is due to the different versions of Python
libraries used in the artifact.
%
Unfortunately, we cannot figure out which versions were used when we conducted
evaluation for the paper.
%
Nonetheless, the soundness percentages produced by the artifact only deviate
slightly from those reported in Table~1.
%
These deviations do not affect the key conclusion of the paper, namely the
effectiveness of the resource-decomposition framework.

In this table, the row for the benchmark \quicksorttiml{} is
\begin{verbatim}
quicksort_timl                N/A                0           100.0%         70.1%
\end{verbatim}
%
It states that the result of Conventional AARA (in the second column of the
table) is non-applicable (N/A).
%
Our evaluation goal of resource decomposition is to empirically demonstrate that
it performs better than each constituent resource-analysis techniques.
%
Since \labelcref{introduction:instantiation:3} integrates TiML and Bayesian
analysis, we do not aim to show that resource decomposition performs better than
Conventional AARA on the benchmark \quicksorttiml{}.
%
Hence, the entry for \quicksorttiml{} in the table shows N/A.
