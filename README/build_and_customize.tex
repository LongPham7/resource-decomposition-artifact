% !TeX root = README.tex

\section{Build and Customize the Artifact}

This section describes how to build and customize Hybrid RaML.

\subsection{Source Code}
\label{sec:Source Code}

Our artifact contains three tools that we have built by ourselves: (i) Hybrid
RaML (in OCaml), (ii) the volesti-RaML interface (in C++), and (iii) the
execution and analysis of benchmark suite (in Python).
%
Their source code is stored in directories \texttt{raml},
\texttt{volesti\_raml\_interface}, and \texttt{benchmark\_suite} inside the
project directory \texttt{/home/hybrid\_aara} in the Docker image.

The project directory contains two additional directories that come from other
sources.
%
The third-party C++ library volesti\footnote{Available here:
  \url{https://github.com/GeomScale/volesti}.} is stored in the directory
\texttt{volesti}.
%
The linear-program solver COIN-OR CLP\footnote{Available here:
  \url{https://github.com/coin-or/Clp}.} is stored in the directory \texttt{clp}.

\subsection{Build the Docker Image}
\label{sec:Build the Docker Image}

The code for building a Docker image is available on GitHub:
\url{https://github.com/LongPham7/hybrid_aara_artifact}.
%
To build a Docker image, clone the GitHub repository and then run (in the root
directory)
\begin{verbatim}
$ docker build -t hybrid_aara .
\end{verbatim}
%
We need a period at the end of the command to indciate that \texttt{Dockerfile}
exists in the current working directory.
%
The build will take 20--30 minutes.
%
The resulting image is named \texttt{hybrid\_aara} and is stored locally on your
machine.

To run the image \texttt{hybrid\_aara}, run
\begin{verbatim}
$ docker run --name hybrid_aara -it --rm hybrid_aara
\end{verbatim}
%
It creates and runs a Docker container with the same name \texttt{hybrid\_aara}.
%
If you want to save the image as a tar archive and compress it, run
\begin{verbatim}
$ docker save hybrid_aara | gzip > hybrid_aara.tar.gz
\end{verbatim}

\subsection{Test Custom Input OCaml Programs}

Suppose we want to analyze a custom OCaml function $f$ using Hybrid AARA.
%
We first prepare two files: (i) an OCaml source file and (ii) a JSON
configuration file.

The OCaml source file stores (i) the OCaml code of the function $f$ (and
possibly other auxiliary functions) and (ii) the OCaml code to be executed in
order to collect runtime cost samples of the function $f$.
%
For example, in the file \texttt{append.ml} described in \cref{sec:Example
  Program}, the OCaml code used for generating runtime cost data has the form
\begin{verbatim}
let input_dataset = [([31; 11; 26; 25], [10; 36; 20; 24]); ...] in
map input_dataset (fun (x, y) -> append x y)
\end{verbatim}
%
This code first creates a list of inputs (i.e. pairs of integer lists) to the
function \texttt{append} and then runs the function on every input.
%
During its execution, whenever we encounter an annotated code fragment
\texttt{Raml.stat(...)}, we record its runtime sample, namely the input, output,
and cost.
%
We aggregate all runtime cost samples into a dataset, which will later be used
in data-driven resource analysis.

For convenience, our artifact comes with a Python script for generating a
collection of inputs for OCaml programs.
%
The file \texttt{input\_data\_generation.py} inside the directory
\texttt{/home/hybrid\_aara/benchmark\_suite/toolbox}
contains a function \texttt{create\_exponential\_lists}, which creates a
collection of integer lists whose sizes grow exponentially (e.g., 1, 2, 4, 8,
etc.).
%
The content of each integer list is chosen randomly.
%
The function \texttt{convert\_list\_python\_to\_ocaml} then converts integer
lists from the Python format to the OCaml format.
%
To see an example, in the directory
\texttt{/home/hybrid\_aara/benchmark\_suite/toolbox}, run
\begin{verbatim}
# python3 input_data_generation.py
\end{verbatim}
%
It prints out a collection of lists whose sizes grow exponentially.

Finally, the JSON configuration file specifies hyperparameters of Hybrid AARA.
%
To learn its syntax, you can take a look at the file \texttt{config.json} inside
each benchmark's \texttt{bin} directory.

\subsection{Modify and Compile Source Code inside the Docker Container}

To modify files inside the Docker container while it runs, we can use the text
editor Vim.
%
If you wish to use a different text editor, modify \texttt{Dockerfile} in the
GitHub repository and rebuild the artifact as instructed in \cref{sec:Source
Code}.
%
Alternatively, while the Docker container runs, you can install the text
editor by
\begin{verbatim}
# apt update && apt install <text-editor> -y
\end{verbatim}

To recompile code \emph{inside} the Docker container after the code has been
modified, follow these steps.
%
For the OCaml source code of Hybrid RaML, go to the directory
\texttt{/home/hybrid\_aara/raml} and then run
\begin{verbatim}
# eval $(opam env)
# make
\end{verbatim}
%
The first line, which initializes OCaml-related environment variables, is only
necessary the first time you recompile Hybrid RaML's source code.

To compile the C++ source code of the volesti-RaML interface, go to the
directory \texttt{/home/hybrid\_aara/volesti\_raml\_interface/build} and then
run
\begin{verbatim}
# cmake .. && cmake --build .
\end{verbatim}

All the other code in the artifact, such as Python scripts inside the benchmark
suite, does not require recompilation after modification.
