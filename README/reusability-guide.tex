% !TeX root = README.tex

\section{Reusability Guide}
\label{sec:reusability-guide}

\subsection{Source Code}

Inside the artifact, under the directory \texttt{/home}, the code that we
contribute is located in two directories: \texttt{ocaml-benchmarks} and
\texttt{experiment}.
%
All the other directories under the directory \texttt{/home} come from existing
tools (with minor modifications).

\paragraph{OCaml}

The directory \texttt{ocaml-benchmarks} contains the OCaml source code of all
benchmarks for all three instantiations of resource decomposition: the 13
benchmarks in \labelcref{introduction:instantiation:1}, the benchmark \kruskal{}
in \labelcref{introduction:instantiation:2}, and the benchmark
\quicksorttiml{}\footnotemark in \labelcref{introduction:instantiation:3}.
%
For each benchmark, its OCaml source code is located in the directory
\begin{verbatim}
/home/ocaml-benchmarks/lib/<benchmark-name>
\end{verbatim}
%
The benchmarks \unbalancedbst{} and \balancedbst{} share the same directory:
\texttt{binary\_search\_tree}.
%
Each benchmark's directory contains an original program (annotated with
\texttt{Raml.tick}), a resource-decomposed program (annotated with
\texttt{Raml.mark}), and a resource-guarded program (if exists).
%
\footnotetext{
  Although \quicksorttiml{} in \labelcref{introduction:instantiation:3} is not
  analyzed by AARA, we have prepared its OCaml source code such that we can
  measure the cost and resource components of \quicksorttiml{} by running its
  OCaml program.
  %
  These measurements are then used to produce a plot of inferred bounds
  (Figure~6 in the paper).}

Additionally, the directory \texttt{ocaml-benchmarks} contains an infrastructure
for running OCaml programs to measure their costs and resource components.
%
The main OCaml source file is \text{main.ml} in the directory
\texttt{ocaml-benchmarks/bin}.
%
Given command-line arguments, the program \texttt{main.ml} runs a specified
benchmark program on specified program inputs.

\paragraph{Python}

The directory \texttt{experiment} contains Python code to
\begin{enumerate*}[label=(\roman*)]
  \item run OCaml programs (in the directory \texttt{ocaml-benchmarks/lib})
        to generate datasets $\calD$ of cost and resource-component measurements;
  \item perform Bayesian inference on the datasets $\calD$; and
  \item produce tables and plots of inferred bounds.
\end{enumerate*}
%
Under the directory \texttt{experiment}, the Python script
\begin{verbatim}
runtime_data_generation/runtime_data_generation.py
\end{verbatim}
generates cost and resource-component measurements.
%
The Python script \texttt{run\_model.py} performs Bayesian inference.
%
The script \texttt{run\_analysis.py} produces tables, and the script
\begin{verbatim}
visualization/visualization.py
\end{verbatim}
produces plots.

\paragraph{Other directories}

The remaining directories under the directory \texttt{/home} come from other
artifacts:
\begin{itemize}
  \item \texttt{raml} is RaML~\citep{Hoffmann2017,RaML};
  \item \texttt{clp} is the linear-program (LP) solver CLP~\citep{CLP} used by
        RaML;
  \item \texttt{timl} is TiML~\citep{WangWC17}; and
  \item \texttt{z3-4.4.1-x64-ubuntu-14.04} is the Z3 SMT
        solver~\citep{DeMoura2008} used by TiML.
\end{itemize}

\subsection{Custom Input Programs for Resource Decomposition}

Suppose you have your own program $P(x)$ that you want to analyze by resource
decomposition.
%
If the resource analysis of the program $P(x)$ uses AARA (e.g.,
\labelcref{introduction:instantiation:1,introduction:instantiation:2}), you create
a new directory for this program:
\begin{verbatim}
/home/ocaml-benchmarks/lib/<benchmark-name>
\end{verbatim}
%
In this directory, you place three OCaml source files for the original program
$P(x)$, the resource-decomposed program $P_{\mathrm{rd}}(x)$, and the
resource-guarded program $P_{\mathrm{rg}}(x, r)$.
%
You also update the main OCaml source file
\begin{verbatim}
/home/ocaml-benchmarks/bin/main.ml
\end{verbatim}
by enabling it to take a command-line argument specifying your custom program.

If the analysis of the program $P(x)$ uses Bayesian inference (e.g.,
\labelcref{introduction:instantiation:1,introduction:instantiation:3}), you
should update the Python code in the directory \texttt{/home/experiment}
so that the relevant Python scripts can take in command-line arguments
specifying your program.
%
For instance, you should extend the following Python file, which lists various
benchmark parameters (e.g., the name of the benchmark, its ground-truth bound,
and the number of resource components):
\begin{verbatim}
/home/experiment/benchmark_data/benchmark_parameters.py
\end{verbatim}

\subsection{Build the Docker Image}

To build this artifact, clone this GitHub repository:
\begin{center}
  \url{https://github.com/LongPham7/resource-decomposition-artifact}.
\end{center}
%
In addition to the code for building the artifact, the repository contains the
\LaTeX{} code for producing this \texttt{README.pdf} document.
%
In the root directory of the repository, run
\begin{verbatim}
$ docker build -t resource-decomposition .
\end{verbatim}
where the period at the end of the command indicates that the current (i.e.,
root) directory contains a Dockerfile.
%
This command builds a Docker image named \texttt{resource-decomposition} and
stores it on your local machine.
%
To run the Docker image, follow the instructions in
\cref{sec:getting-started-guide}.

To save the Docker image in an \texttt{.tar.gz} file, run
\begin{verbatim}
$ docker save resource-decomposition | gzip > resource-decomposition.tar.gz
\end{verbatim}
