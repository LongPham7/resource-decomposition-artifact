% !TeX root=README.tex

\section{Introduction}

\paragraph{What is included}

From Zenodo, you should obtain the following items:
\begin{itemize}
  \item This \texttt{README.pdf} document; and
  \item Archived and compressed Docker image
        \texttt{resource-decomposition.tar.gz}.
\end{itemize}

\paragraph{Artifact overview}

\emph{Hybrid resource analysis} integrates two resource-analysis techniques with
complementary strengths and weaknesses.
%
Given a program, the goal of resource analysis is to infer its worst-case
symbolic bound on resource usage (e.g., running time and memory).
%
Three existing families of resource-analysis techniques, namely static,
data-driven, and interactive analyses, complement each other.
%
For example, static resource analysis is sound but incomplete due to the
undecidability of resource analysis in general.
%
By contrast, data-driven resource analysis, which statistically infers a cost
bound from a dataset of cost measurements, is complete but unsound.
%
Additionally, compared to static and data-driven analyses, interactive resource
analysis, where human intervention is required, is not fully automated but has
higher expressiveness and more reasoning power.
%
By integrating two analysis techniques, hybrid resource analysis retains their
strengths while mitigating their respective weaknesses.

\emph{Resource decomposition} is a new approach to hybrid resource analysis
proposed in our paper.
%
It works as follows.
%
First, given an input program $P(x)$, the user annotates the source code to
specify a quantity called \emph{resource component} (e.g., the recursion depth
of a helper function inside the program $P(x)$).
%
The resulting annotated program $P_{\mathrm{rd}}(x)$ is called a
\emph{resource-decomposed program}.
%
The first resource analysis is performed on the resource-decomposed program
$P_{\mathrm{rd}}(x)$, inferring a symbolic bound $g(x)$ of the resource
component.
%
Meanwhile, the original program $P(x)$ is extended with a numeric variable $r$,
called a resource guard, which tracks the user-specified resource component.
%
The resulting program $P_{\mathrm{rd}}(x, r)$ is called a \emph{resource-guarded
  program}.
%
The second resource analysis is performed on the resource-guarded program to
infer its overall cost bound $f(x, r)$.
%
Finally, an overall cost bound of the original program $P(x)$ is obtained by
composing the two inference results.
%
That is, we substitute the resource-component bound $r \leq g(x)$ for the
resource guard $r$ in the bound $f(x, r)$, yielding $f(x, g(x))$.

Our artifact contains code for evaluating three concrete instantiations of the
resource-decomposition technique (\S4--6 in the paper).
%
These instantiations each combine the following pairs of resource-analysis
techniques:
\begin{enumerate}[label={Inst. \arabic*}]
  \item Static analysis (AARA~\citep{Hoffmann2011a,Hoffmann2017}) and Bayesian
        data-driven analysis (\S4);
        \label{introduction:instantiation:1}
  \item Static analysis (AARA~\citep{Hoffmann2011a,Hoffmann2017}) and
        proof-assistant-based interactive resource analysis (Iris with time
        credits~\citep{Chargueraud2019}) (\S5); and
        \label{introduction:instantiation:2}
  \item SMT-based semi-automatic analysis (TiML~\citep{WangWC17}) and Bayesian
        data-driven analysis (\S6).
        \label{introduction:instantiation:3}
\end{enumerate}
%
\labelcref{introduction:instantiation:1} is evaluated on 13 benchmarks listed in
Table~1 of the paper.
%
\labelcref{introduction:instantiation:2} is evaluated on Kruskal's algorithm
(\kruskal{}) for minimum spanning trees.
%
\labelcref{introduction:instantiation:3} is evaluated on a version of quicksort
(\quicksorttiml{}) where the cost of integer comparison is logarithmic in the
maximum input number (i.e., linear cost in the number of bits).

Out of the six analyses (i.e., two resource analyses in each instantiation), the
artifact performs five of them: all except the interactive resource analysis in
the second instantiation (\S5).
%
The inference result of the interactive analysis is available in
\citet{Chargueraud2019} and hence is not included in our artifact.

\paragraph{Supported claims}

The artifact supports the following claims in the paper:
\begin{itemize}
  \item \labelcref{introduction:instantiation:1}: Percentages of sound bounds
        and analysis time, which are listed in the last four columns of Table~1.
  \item \labelcref{introduction:instantiation:1}: Plots of posterior
        distributions of the 13 benchmarks (Figures~3,4, 7--20).
  \item \labelcref{introduction:instantiation:2}: Cost bound inferred by static
        analysis AARA~\citep{Hoffmann2011a,Hoffmann2017} for the
        resource-guarded program of Kruskal's algorithm.
        %
        The bound is displayed in Eq.~(5.3) in the paper.
  \item \labelcref{introduction:instantiation:2}: Plots of inferred symbolic
        bounds (Figure~5).
  \item \labelcref{introduction:instantiation:3}: Cost bound inferred by
        TiML~\citep{WangWC17} for the benchmark \quicksorttiml{}, which is
        displayed in Eq.~(6.1) in the paper.
  \item \labelcref{introduction:instantiation:3}: Plots of inferred symbolic
        bounds (Figure~6).
\end{itemize}
